\documentclass[a4paper,12pt]{article}
\usepackage[T1, T2A]{fontenc}
\usepackage[utf8]{inputenc}
\usepackage[russian,english]{babel}
\usepackage{graphicx}
\usepackage{multicol}
\usepackage{fancyhdr}
\usepackage{geometry}
\usepackage{titlesec}
\usepackage{amsmath}
\sloppy

\geometry{top=2cm,bottom=2cm,left=2cm,right=2cm}
% Document begins
\begin{document}
\begin{large}

\setlength{\parindent}{20pt} % Исправленная команда для отступа абзаца
\section{Вербальные и невербальные способы взаимодействия с аудиторией во время публичного выступления.}

\hspace{\parindent}Люди имеют неоспоримое преимущество перед другими формами жизни: они умеют общаться. 
Общение считается одной из главных форм социальной активности человека. 

 В процессе общения то, что раньше знал и умел один человек, становится достоянием массы
 людей. 
 Общение в научном понимании представляет собой взаимодействие людей и обмен
 информацией при этом взаимодействии.

 Выделяют две группы способов, которыми может осуществляться взаимодействие между людьми: \textbf{вербальные} и \textbf{невербальные} средства общения. Вербальное общение осуществляется с помощью слов. К невербальным средствам общения относятся позы, жесты, мимика, направление и продолжительность взгляда, расположение собеседников в пространстве и расстояние между ними, громкость, тон и тембр голоса, интонация,
 артикуляция и прочее.
 
 Разделение \textbf{словесной} и \textbf{несловесной} сторон речи очень условно и возможно только для удобства описания, поскольку и вербальная, и невербальная стороны общения очень редко существуют друг без друга.
 
 Речевая деятельность представляет собой процесс и является самым распространенным и самым сложным видом человеческой деятельности. На две трети человеческая деятельность
 состоит из речевой.

 Существуют четыре вида речевой деятельности (\textbf{аудирование, говорение, чтение и письмо}) и три основных жанра устного речевого общения (\textbf{беседа}; \textbf{разговор}, отличающийся от беседы целенаправленностью;\textbf{спор}).
\end{large}
\end{document}
