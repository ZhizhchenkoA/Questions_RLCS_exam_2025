\section{Вербальные и невербальные способы взаимодействия с аудиторией во время публичного выступления.}

\hspace{\parindent}Люди имеют неоспоримое преимущество перед другими формами жизни: они умеют общаться. 
Общение считается одной из главных форм социальной активности человека. 

 В процессе общения то, что раньше знал и умел один человек, становится достоянием массы
 людей. 
 Общение в научном понимании представляет собой взаимодействие людей и обмен
 информацией при этом взаимодействии.

 Выделяют две группы способов, которыми может осуществляться взаимодействие между людьми: \textbf{вербальные} и \textbf{невербальные} средства общения. Вербальное общение осуществляется с помощью слов. К невербальным средствам общения относятся позы, жесты, мимика, направление и продолжительность взгляда, расположение собеседников в пространстве и расстояние между ними, громкость, тон и тембр голоса, интонация,
 артикуляция и прочее.
 
 Разделение \textbf{словесной} и \textbf{несловесной} сторон речи очень условно и возможно только для удобства описания, поскольку и вербальная, и невербальная стороны общения очень редко существуют друг без друга.
 
 Речевая деятельность представляет собой процесс и является самым распространенным и самым сложным видом человеческой деятельности. На две трети человеческая деятельность
 состоит из речевой.

 Существуют четыре вида речевой деятельности (\textbf{аудирование, говорение, чтение и письмо}) и три основных жанра устного речевого общения (\textbf{беседа}; \textbf{разговор}, отличающийся от беседы целенаправленностью; \textbf{спор}).

 В результате системных исследований, социологических опросов, контекстного анализа
 корпуса литературных произведений была выделена параметрическая модель русского коммуникативного поведения, объединяющая его вербальные и невербальные характеристики:
 \begin{itemize} 
    \item высокая степень общительности, эмоциональность, искренность;
    \item приоритетность неформального общения;
    \item низкое внимание при восприятии речи собеседника;
    \item пониженный уровень вежливости;
    \item высокая регулятивность общения;
    \item менторская доминантность;
    \item высокая степень бескомпромиссности;
    \item широта и интимность обсуждаемой информации;
    \item коммуникативный пессимизм;
    \item бытовая неулыбчивость;
    \item короткая дистанция общения и допустимость физического контакта;
    \item коммуникативный эгоцентризм;
    \item пониженный коммуникативный самоконтроль.
\end{itemize}

Если в общении принимают участие несколько человек — тогда их общение близко к межличностному (при небольшом количестве общающихся их взаимодействие называют групповым), а может быть и человек 20—50, и в этом случае оно становится безусловно публичным даже в неофициальной обстановке. Групповое общение характеризуется тем, что в нем еще возможен диалог (если не между всеми, то со многими), но в таком общении уже необходим лидер, который будет регулировать это общение.

 Публичное общение обычно протекает в форме \textbf{монолога}. Оно всегда требует структурирования, поскольку люди в таких случаях собираются вместе ради достижения какой-то важной цели. Без структурной организации общения эта цель вряд ли будет достигнута. При публичном общении возникает иная, более высокая степень ответственности за речь, и одним из главных требований к ней становится целенаправленность и содержательность. Возрастает в этом случае и уровень требований к оформлению речи, к соблюдению в ней этических и коммуникативных норм, к ее правильности и эстетичности.

 Для вербального взаимодействия с аудиторией необходимо владеть ораторским красноречием, а именно умениями:
\begin{itemize}
    \item найти что сказать;
    \item найденное расположить по порядку;
    \item придать ему словесную форму;
    \item утвердить все это в памяти;
    \item расположить к себе слушателей;
    \item изложить суть дела;
    \item подкрепить свое положение;
    \item опровергнуть мнение оппонента;
    \item придать блеск своим положениям;
    \item окончательно низвергнуть противника.
\end{itemize}

 Невербальное взаимодействие с аудиторией проявляется в паралингвистических средствах. 
 Различаются три вида паралингвистических средств: \textbf{фонационные, кинетические и графические} (в письменной речи). 
 К \textit{фонационным} паралингвистическим средствам
 относятся: \textbf{тембр голоса, темп и громкость речи, устойчивые интонации, особенности
 произнесения звуков, заполнения пауз} (\textit{э, мэ...}). К \textit{кинетическим} компонентам речи относятся: \textbf{жесты, позы, мимика}.

 В русской коммуникативной среде существует определенный набор правил, демонстрирующих уважение к спутнику или собеседнику:
 \begin{itemize}
    \item подать/ помочь снять женщине пальто;
    \item пропустить женщину при входе в дверь, первому зайти в лифт;
    \item идти впереди женщины по лестнице;
    \item подать руку женщине при выходе из транспорта;
    \item встать, когда встает / входит дама, уважаемый или пожилой человек;
    \item уступить место старшему;
    \item проводить гостя до двери, до выхода;
    \item снять солнцезащитные очки при разговоре;
    \item смотреть собеседнику в лицо;
    \item говорить ровным тоном, не повышая голоса.
\end{itemize}

 Однако при личном общении люди не так ярко жестикулируют или выражают эмоции (при обычном диалоге). 
 При общении с аудиторией говорящему необходимо \textit{драматизировать} собственную речь, то есть показывать эмоции ярче и пользоваться паралингвистическими
 средствами практически на сто процентов.
 
 Более того одним из современных направлений лингвистических исследований является
 \textit{этнопсихолингвистика} - наука, изучающая психолингвистические типы представителей разных этносов. Она сложилась на стыке психологии, лингвистики, социологии и культурологии.
 Этнопсихолингвистика вместе с антрополингвистикой изучает языковые картины мира представителей разных этнических сообществ.
 
 Каждая нация обладает определенным набором психологических и поведенческих стереотипов, в той или иной мере присущих всем членам данного исторически сложившегося
 социума. 
 
 Этнопсихолингвистические исследования показывают, что нормы и особенности речевого поведения, коммуникативные стереотипы также обладают национально-культурной
 спецификой. Они исторически сложились в той или иной культуре и отражают ее систему
 ценностей.
 
 Культура может быть средством как общения, так и разобщения людей, и немалую роль в этом играет язык.
 Понятие этноса, формированию которого способствуют
 природно-географические, социальные и исторические факторы, соотносится с понятием локальной культуры. Следует помнить, что всем народам присущ этноцентризм, поэтому сопряжение моделей восприятия действительности не может быть бесконфликтным.
 
 В рамках межкультурного общения выделяются лакуны - несовпадения образов сознания участников коммуникации, пробелы, белые пятна на семантической карте языка, текста и
 культуры. Лакуны незаметны для носителей языка, но выявляются при сопоставлении с речью носителей других языков в условиях диалога. При выборе тактик понимания чужой культуры необходимо разрабатывать стратегии совмещения своего и чужого.

