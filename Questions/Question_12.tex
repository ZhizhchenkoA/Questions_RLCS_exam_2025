\section{Коммуникация и общение. Функции и цели общения.}

\ \ Коммуникативный шок возникает при неожиданном соприкосновении с незнакомым, чуждым речевым поведением. Для устранения коммуникативного шока необходимо системное описание коммуникативного поведения, которое можно разделить на три вида: личностное, групповое и национальное. В современном русскоязычном обществе выделяются две социальные группы: носители литературного языка и носители просторечия(между ними существуют двуязычные прослойки). В условиях формирования глобального коммуникативного пространства необходимо нивелировать, сглаживать национально специфические особенности коммуникативного поведения.

\subsection*{Параметрическая модель коммуникативного поведения}
 

Параметрическая модель коммуникативного поведения включает в себя ряд характеристик, в том числе следующие.

\begin{itemize}
\item{\bfseries Контактность} (в русской коммуникативной структуре допустим физический контакт. Очень распространен ритуал рукопожатия (жест преимущественно мужской). Инициирует рукопожатие лицо более высокого ранга или женщина. В русской коммуникативной традиции можно деликатно дотрагиваться до лиц противоположного пола, в учебном процессе --- до учащегося. Это рассматривается как дружеское расположение, а не как приставание).
\item {\bfseries
Неформальность \textmd{(русские демократичны в общении, любят общаться без церемоний; вторгаются в личную сферу собеседника, как им кажется, из лучших побуждений. Свободно обсуждаются запретные темы: политика, религия, профессиональные навыки конкурентов и их личная жизнь, доходы, возраст, вес)}}.
\item {\bfseries
Самопрезентация} (в общении отмечается стремление установить {{\bfseries коммуникативную доминантность}. В компании русский человек любит  "блеснуть", сказать  "последнее лово" в споре. Отмечается свобода подключения к общению)}.
\item {\bfseries
Пониженная вежливость \textmd{(императивность русских этикетных норм(те обязательность их применения) пониженная. По меткому замечанию Н.В.Гоголя, русский человек не любит признаваться в своих ошибках и с трудом отвечает на вопросы. Временами русские грубы, особенно в телефонных беседах; мат в общении не пересекается. Но внимание к старшему поколению выше, чем на Западе)}}.
\item {\bfseries
Регулятивность \textmd{(с точки зрения западноевропейцев, русские постоянно вмешиваются в дела других и пытаются регулировать их поведение. Русский человек может открыто предъявлять претензии незнакомым, часто высказывает посторонним замечания. В русском сознании }\textmd{считается, что плохо, когда "никому ни до чего нет дела". В русской коммуникативной традиции принято предупреждать незнакомых о возможных
неприятностях).}}
\item {\bfseries
Конфликтность \textmd{(русские любят спорить и в ходе спора проявляют бескомпромиссность. Русские совершенно нетерпимы к чужому мнению. Они категоричны, любят критиковать других, способны к самокритике, но, если их самих критикуют иностранцы, то сразу горячо протестуют. Бесконфликтное общение воспринимается как неумение постоять за себя и не поощряется)}}.
\item {\bfseries
Откровенность \textmd{(в русской коммуникативной среде приветствуется откровенный разговор (разговор по душам). Осуждается человек, уклоняющийся от задушевной беседы (не наш!). Для русской коммуникативной традиции характерно негативное отношение к светскому общению ("ненастоящему"), к разговору на общие темы. Русским свойственно любопытство и стремление к широте обсуждаемой информации. Оценочность в общении, категоричность).}}
\item {\bfseries
Коммуникативный пессимизм \textmd{(русские любят жаловаться на плохую жизнь и задают слишком много вопросов, при этом уровень их интеррогативности очень высок).}}
\item {\bfseries
Коммуникативный эгоцетризм \textmd{(русских отличает переключение внимания на себя в любом разговоре. Похвала используется редко. Русские не умеют говорить комплименты, стесняются благодарить за них)}}.
\item {\bfseries
Сверхкраткая дистанция \textmd{(русские слишком близко подходят к собеседнику и садятся вплотную друг к другу, очень терпеливы к давке в толпе)}}.
\item {\bfseries
Нерегламентированность общения \textmd{(русский диалог может идти долго. Русские могут перебивать собеседника, не склонны скрывать отсутствие интереса к теме беседы)}}.
\item {\bfseries
Пониженный самоконтроль\textmd{ (отношение к собственным речевым ошибкам в русском коммуникативном сознании снисходительное: не принято следить за правильностью своей речи, зато не возбраняется указывать собеседнику на его ошибки)}}.
\end{itemize}

\subsection*{Паралингвистические средства общения (невербальные)}
 

Иностранцы подмечают специфические русские невербальные сигналы, не встречающиеся в западных коммуникативных культура: \textit{чесать рукой ухо} --- решать сложную проблему; \textit{показывать кукиш}
--- выражать категорический отказ; \textit{держать себя за горло} --- подчеркивать стесненные обстоятельства; \textit{щелкать по горлу} --- приглашать выпить.
Наиболее распространенными русскими невербальными сигналами являются \textit{кивок, поворот головы, покачивание головой, пожимание плечами.}

{\bfseries
\textmd{Иностранцы отмечают у русских малую дистанцию в общении, стремление к физическому контакту, интенсивную жестикуляцию, высокую ее амплитуду. Русской жестикуляции свойственна тенденция к асимметрии, то есть производится мало жестов обеими руками, в основном, участвует правая рука. Взгляд в русской коммуникативной среде несет большую эмоциональную }\textmd{нагрузку. Улыбка русских весьма своеобразна. Она выполняет совершенно иные функции, чем в европейской культуре, и не является сигналом вежливости. Не \ принято улыбаться незнакомы, потому что улыбка
- сигнал личного расположения. Таким образом, выделяются следующие доминантные черты русского коммуникативного поведения:}}

\begin{itemize}
\item {\bfseries
\textmd{высокая степень общительности, эмоциональность, искренность}}
\item {\bfseries
\textmd{пониженная императивность этикетных норм}}
\item {\bfseries
\textmd{низкое внимание при восприятии речи собеседника}}
\item {\bfseries
\textmd{пониженный уровень вежливости}}
\item {\bfseries
\textmd{высокая регулятивность общения}}
\item {\bfseries
\textmd{менторскся доминантность}}
\item {\bfseries
\textmd{высокая бескомпромиссность}}
\item {\bfseries
\textmd{приоритетность разговора по душам}}
\item {\bfseries
\textmd{широта обсуждаемой информации}}
\item {\bfseries
\textmd{коммуникативный пессимизм}}
\item {\bfseries
\textmd{бытовая неулыбчивость}}
\item {\bfseries
\textmd{короткая дистанция общения и допустимость физического контакта}}
\item {\bfseries
\textmd{коммуникативный эгоцентризм}}
\item {\bfseries
\textmd{пониженный коммуникативный самоконтроль}}
\item {\bfseries
\textmd{устойчивое пренебрежение интересами окружающих}}
\end{itemize}

\subsection*{Коммуникативный эталон\textmd{ русских}}
 

Образцом считается человек, который умеет хорошо слушать и вовремя дать совет, который способен убедить
собеседника и прийти к консенсусу, не навязывающий свою точку зрения, образованный, эрудированный, дружелюбный,
откровенный, сдержанный, вежливый, оптимист, с чувством юмора, хорошими манерами и опрятный.

\subsection*{Коммуникативные неудачи}
 

Коммуникативные неудачи постоянны в общении людей, они естественны и часто приводят к недопониманию.
Этноцентризм --- свойство почти всех культур. Коммуникативные неудачи классифицируются по разным основаниям: социально-культурным, психосоциальным и языковым. К коммуникативным неудачам нередко приводят различия в речевых стратегиях говорящего. Нарушение норм национального специфического речевого поведения воспринимается как непроизвольное вторжение в интимную сферу. Коммуникативные неудачи связаны с недостаточным знанием не только языка, но и обычаев другого народа. Так в Китае суп подают после еды; не зная, что это означает завершение трапезы, иностранцы могут затянуть свой визит, в ожидании продолжения. Коммуникативные неудачи могут быть связаны с невербальными средствами общения. В январе 2005 г. в европейской прессе прошло сообщение: общественность Норвегии была шокирована тем, что во время инаугурации президент США Джордж Буш сделал жест, который у норвежцев считается приветствием дьяволу (выставленный вперед указательный палец и мизинец). Коммуникативное поведение - это совокупность норм и традиций общения в определенном лингвокультурном сообществе.
В русском общении меньше норм и больше традиций.
Поэтому русскому человеку легче овладеть высоконормированной западной моделью, чем западному человеку освоить нечетко очерченные традиции русского общения, являющиеся отражением специфики русской культуры, которую Ю. М. Лотман определил как "бинарную", развивающуюся путем взрывов глобального характера. Культурные расколы, разломы отражаются в размытости норм русского коммуникативного поведения.

\subsection*{Общение. Что представляет собой общение.}
 

Общение пронизывает все сферы деятельности человека, поэтому изучается многими науками, в рамках каждой из
которых ученые приходят к феномену общения со своих позиций. Социологи, психологи, философы, лингвисты, понимают под
общением "процесс выработки новой информации и то, вырабатывает их
общность" (М.С.Каган) или "особую форму взаимодействия
людей" и тд

Иначе и быть не может. Потому что, прежде всего,"общим для людей является предстоящая им действительность"(Н.И.Жинкин). Эта общая действительность включает в себя  общий язык, и общую память, и общие понятия, и общие механизмы мышления, выработанные в результате общей культуры, и другое, благодаря чему общение становится не только необходимым, но и возможным.

Психологи выделяют три уровня анализа структуры общения:

\begin{enumerate}
    \item Общение индивидуума как сторона его образа жизни(макроуровень).
    \item Отдельные акты общения, отдельные контакты (беседа, спор и пр.) --- мезоуровень.
    \item Отдельные элементы акта общения(средства выражения) --- микроуровень.
\end{enumerate}

Мы рассматриваем общение на макроуровне.
Д.С.Лихачев "Общаясь, люди создают друг
друга".
В этих словах подчеркивается именно взаимность влияния людей друг на друга и необходимость их друг другу для создания своей индивидуальности и для развития каждого.
Такой подход и такое понимание подчеркивает обусловленность общения как основного элемента культуры.
Это объясняет основное средство общения --- речь.
Общение --- это реальная деятельность, разворачивающаяся процессуально, то есть так же, как и речь, и протекающая преимущественно в виде речи (в ее словесной и несловесной составляющих).
При этом деятельностью является не только то общение, которое происходит при совместном решении каких-либо предметно-практических задач, но и духовное общение, в ходе которого происходит духовно-информационное взаимодействие. Общение как деятельность требует сознательного целеполагания, выбора оптимальных средств для достижения поставленных целей, постоянного слежения за действиями партнеров и внесения каждым необходимых коррективов в собственное поведение и, конечно же, ответственности за результат этой деятельности.

\subsection*{Основные функции общения }
 

А. А. Брудный выделяет 4 основные функции общения:

\begin{enumerate}
    \item Инструментальную, то есть общение как вспомогательный компонент совместной предметной деятельности (например, ремонта машины или уборки).
    \item Синдикативную (объединения), когда общение предполагает создание единства вступивших в него участников.
    \item Функцию самовыражения, которая по своей сущности ориентирована на взаимопонимание, на контакт.
    \item Трансляционную функцию - передачи конкретных способов деятельности, оценочных критериев и программ (например, обучение).
\end{enumerate}

{\bfseries
\textmd{Значит, для реализации любой функции общения необходим субъект, который осуществляет это общение. В свою очередь в рамках общения как деятельности и сам человек как субъект общения также выполняет разнообразные функции: коммуникативную (обеспечение взаимосвязи), информационную (взаимовыражение), когнитивную (взаимопознание), эмотивную(переживание взаимоотношение), конативную (взаимопроявление, управление), креативную (взаимовлияние, преобразование). Таким образом, общение удовлетворяет различные потребности личности. Эти потребности свойственны всем людям, поскольку человек - создание прежде всего социальное. Все эти потребности носят культурный в своей основе характер: они связаны с ценностями и ценностными отношениями к себе и другим, с диалогичностью, с процессами познания и самопознания, с творческой деятельностью и т. д. Но, в зависимости от уровня культуры конкретного человека, те или иные социальные потребности получают первостепенное значение. Общение - это одно из основных условий существования культуры. Оно органично культуре и в том, что в процессе взаимодействия, так же как в культуре в целом, происходит столкновение противоречивых тенденций между объединением и обособлением, социализацией и индивидуализацией, что также становится движущей силой развития и обогащения всех участников общения.}}

\subsection*{Основные цели общения }
 

Основные цели общения связаны с направленностью и особенностями взаимодействия между коммуникантами. Философ М. С. Каган предложил такую классификацию целей общения:

\begin{enumerate}
    \item цель общения находится вне самого взаимодействия субъектов;
    \item цель общения заключена в нем самом;
    \item цель общения состоит в приобщении партнера к опыту и ценностям инициатора общения;
    \item целью общения является приобщение самого его инициатора к ценностям партнера.
\end{enumerate}

Первая цель решается преимущественно в процессе совместных действий партнеров по общению.
Вторая заключается, главным образом, в самопознании и самовыражении путем диалогической деятельности и во взаимопонимании участников общения.
Третья и четвертая цели говорят сами за себя --- это прежде всего ценности взаимодействие партнеров, при котором один из них берет на себя роль инициатора.
Все эти цели достигаются только в процессе диалога, потому, по словам М. М. Бахтина, "в процессе реальной речевой деятельности люди становятся "речевыми субъектами", а их словесное взаимодействие --- не обменом монологов, а диалогом, то есть ориентированными друг на друга высказываниями".
При этом диалог понимается не только как форма речи, предполагающая смену речевых субъектов, а широко, то есть как столкновение, взаимодействие разных точек зрения, разных позиций, разных умов, разных пониманий, разных интерпретаций и~т.~д.

\subsection*{Различие общения и коммуникации}
 

Диалог, с одной стороны, и способы его воплощения, с другой, в реальном общении могут существенно различаться.
Это различие во многом отражено в оттенках значений, которые имеют слова "общение" и "коммуникация".
\textbf{Коммуникация} --- это информационная связь субъекта с тем или иным объектом.
Речевая коммуникация --- это одно из значений, которые в современном русском языке есть у слова "коммуникация".
Коммуникацией называют и пути сообщения (водная коммуникация), и формы связи (телефон, телеграф, радио), и общение, связь между людьми для передачи и получения информации с помощью технических средств --- средств массовой коммуникации (печать, радио, кинематограф, телевидение) численно большим рассредоточенным аудиториям.
\textbf{Общение} (по словарю Ожегова) --- это взаимные отношения, деловая или дружеская связь. 

Эти понятия объединяет, делает синонимами речь, которая связывает людей и служит основным средством передачи информации в различных её формах и видах. Поэтому мы будем рассматривать эти понятия только применительно к речи, к речевой ситуации.

В этом смысле под "\textbf{коммуникацией}" понимается передача речевой
информации от отправителя к получателю и прием этой информации получателем от отправителя. А под
"\textbf{общением}" - речевое взаимодействие между людьми.

\begin{enumerate}
    \item "\textbf{Коммуникация}" в своем прямом значении является исключительно информационным процессом, адресованным человеку, животному, машине (может осуществляться на искусственных языках), а "\textbf{общение}" всегда двухслойно (оно имеет и практический, и духовный(информационный) характер).
    \item "\textbf{Коммуникация}" предполагает информационную связь субъекта с тем или иным объектом. При этом в роли объекта может выступать как человек, так и животное или машина. "Общение" же возможно только между субъектами, то есть между людьми, ощущающими свою индивидуальность и уникальность.
    \item "Коммуникация" --- это прежде всего процесс передачи информации. В этом отношении он односторонен и монологичен. "Общение" - это процесс взаимодействия, он двусторонен и диалогичен.(по М. С. Кагану "Мир общения")
\end{enumerate}

Таким образом, в основе различия оттенков значения слов "общение" и
"коммуникация" лежат особенности отношений между участниками этого процесса.

