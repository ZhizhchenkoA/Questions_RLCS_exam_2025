\section{Публичная речь. Особенности публичной речи.}

\textbf{Особенности публичной речи:}
\begin{itemize}
	\item \textbf{Обратная связь:} Оратор должен наблюдать за
	поведением аудитории, корректировать собственную речь, то есть устанавли­вать контакт со слушателем.
	
	\item \textbf{Устная форма:} Публичная речь реализуется в
	устной форме литературного языка. Для оратора важно так построить публичное выступление, чтобы содержание его речи было понятно слушателям.
	
	\item \textbf{Сложная структура:} Ораторская речь тщательно готовится. Оратор должен не просто механически прочитать текст, а именно произнести его. В процессе импровизации, и появляются элементы устной речи. 
	
	\item \textbf{Коммуникационные средства:} Для воздействия на аудиторию оратор использует не только слова, но и паралингвистические средства — жесты, мимику, интонацию, темп и тембр голоса, что усиливает эмоциональное восприятие речи.
\end{itemize}

\textbf{Оратор} — это мастер красноречия, искусно владеющий словом и способный влиять на мысли и чувства слушателей. Само слово «оратор» происходит от латинского глагола <<\textit{orare}>>, что означает «говорить».

\textbf{Виды красноречия:}
\begin{itemize}
	\item Социально-политическое
	\item Академическое
	\item Судебное
	\item Социально-бытовое:
	\item Духовное
\end{itemize}

\textbf{Типы ораторов:}
\begin{itemize}
	\item \textbf{Логический тип:} Оратор, который строит свою речь на аргументах, фактах и логических построениях, акцентируя внимание на рациональной стороне выступления.
	
	\item \textbf{Эмоциональный тип:} Оратор, основной упор которого — пробуждение чувств и переживаний аудитории, воздействие через эмоции.
	
	\item Идеал: Сочетание логики и эмоций — гармоничный баланс, который позволяет убедительно донести информацию и вызвать нужный отклик у слушателей.
\end{itemize}

\textbf{Качества и навыки оратора:}
\begin{itemize}
	\item Эрудиция и нравственность: Широкий кругозор, знания и высокая моральная позиция, которые обеспечивают уважение и доверие аудитории.
	
	\item Подбор литературы и составление плана: Тщательная подготовка речи, включающая подбор материалов и создание чёткой структуры выступления.
	
	\item Самообладание и ориентирование во времени: Умение сохранять спокойствие, управлять эмоциями и соблюдать регламент выступления.
	
	\item Владение техникой речи: Контроль дыхания, работа с голосом и дикцией для чёткой и выразительной передачи мыслей.
	
	\item Постоянная практика и самообразование: Регулярные тренировки и стремление к совершенствованию мастерства.
\end{itemize}

\textbf{Мотивы слушателей:}
\begin{itemize}
	\item интеллектуально-познавательные мотивы
	\item мотивы морального плана - обязаны присутствовать
	\item мотивы эмоционально-эстетического плана - нравится оратор
\end{itemize}

Оратору необходимо выявить основной мотив, объединяющий данную аудиторию, и соответствующим образом построить свою речь.

\textbf{Виды речей:}
\begin{itemize}
	\item \textbf{Информационные:} Направлены на передачу фактов и знаний.
	\item \textbf{Убеждающие:} Цель — склонить слушателей к определённой точке зрения или действию.
	\item \textbf{Воодушевляющие:} Стимулируют мотивацию, вдохновляют на достижение целей.
	\item \textbf{Призывные:} Взывают к активности, побуждают к конкретным действиям.
	\item \textbf{Развлекательные:} Предназначены для отдыха и эмоционального подъёма аудитории.
\end{itemize}