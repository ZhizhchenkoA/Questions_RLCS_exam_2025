\section{Коммуникативная ситуация. Компоненты коммуникативной ситуации, их взаимосвязь.}
 

Коммуникативные и этические нормы представляют собой конкретные правила, которые помогают осуществить оптимальное общение, то есть такое взаимодействие, которое создает наилучшие условия для выработки и реализации не противоречащих друг другу коммуникативных целей всех партнеров по общению, для создания благоприятного эмоционального климата вследствие преодоления различного рода барьеров, а также для максимального раскрытия личности каждого.

Основное общение происходит на уровне коммуникативной ситуации. Почему ситуации, а не высказывания, не текста?
Потому что высказывание, даже полное и развернутое, --- это только одна из единиц ситуации общения.
Текст сам по себе не дает представления о диалоге, который неизбежно возникает в общении и ради которого общение и происходит.
Кроме того, иногда полностью понять мотивы и цели речи общающихся можно, только рассматривая ситуацию в целом.
А целый ряд личностно ориентированных жанров (личное письмо, дружеский разговор и~т.~д.) невозможно проанализировать и правильно интерпретировать, не учитывая индивидуальности участников общения, что тоже неосуществимо вне конкретной коммуникативной ситуации. Знать все условия ситуации необходимо и тогда, когда без этого знания сложно реконструировать смысловое наполнение свернутой и неконкретной речи.
Вне ситуации общения невозможно также проанализировать и невербальную сторону речи — ее подтекст, смысл пауз или молчания, жестово-мимическое оформление и~т.~д.
Помимо этого, при изолированном рассмотрении текста вне ситуации, вне окружающих высказываний теряется и взаимосвязь текстов, становятся неясными их взаимоотношения, возникают трудности при оценке эффективности текстов и ситуации в целом.

Таким образом, культуру речевой деятельности в конкретной речи (особенно в устной) можно полноценно рассмотреть только в контексте коммуникативной ситуации, тем более, что именно на этом уровне и проявляется в большей степени культура речи.
В коммуникативной ситуации формируются мотивы и цели речи каждого из участников общения и для их реализации осуществляется осознанный выбор наилучшего варианта использования языковых или речевых средств из всех возможных. Важно подчеркнуть, что этот отбор в рамках культуры речи осуществляется на основе соблюдения этических и коммуникативных норм.
Кроме того, результат речи также проявляется в коммуникативной ситуации: коммуникативный успех или коммуникативный провал (неудача) --- это понятия, которые вне конкретной ситуации общения не могут быть адекватно раскрыты.
Безусловно, особенности каждой речевой ситуации и многообразие форм проявления в речи различных факторов, влияющих на ее компоненты, не позволяют решить проблему полного описания каждой ситуации общения.
В то же время существуют общие подходы к рассмотрению, анализу и описанию различных компонентов коммуникативной ситуации, которые помогают применить их к конкретным ситуациям общения.

Содержание понятия «коммуникативная ситуация» распространяется и на ситуации общения, и на ситуации коммуникации, оно включает в себя огромное коммуникативное богатство — это многообразие мотивов и целей, речевых стратегий и тактик, определенных «сценариев» речи, ее форм и жанров, языковых и речевых средств и~т.~д.
Это богатство так велико, что даже перечислить можно только основные виды каждого из этих компонентов коммуникативного богатства, создать реестр только типовых ситуаций общения в той или иной сфере деятельности.
В свою очередь, сочетания всех этих элементов общения неисчерпаемы, что определяет прелесть общения как творчества и трудности, из
этого проистекающие.

Для того чтобы грамотно общаться, необходимо знать основ
ные компоненты коммуникативной ситуации, а также основные
законы общения, которые определяют его коммуникативные и
этические нормы.

\subsection*{Компоненты коммуникативной ситуации.}

Основными компонентами коммуникативной ситуации являются: отправитель и получатель информации, цели каждого, сама информация (предмет речи), форма ее преподнесения и условия общения, благодаря которым появляется возможность взаимодействия этих коммуникантов.

На успешность общения влияет много факторов различного
порядка, таких как:

\begin{itemize}
    \item внеязыковые (экстралингвистические) факторы: взаимоотношения между участниками, этические установки коммуникантов; характер передаваемой информации, условия общения, национально-культурные традиции и т.д.;
    \item языковые и речевые факторы: специфика языка, на котором происходит общение; устная или письменная, монологическая или диалогическая форма сообщения, особенности жанра и стиля речи, степень реализации коммуникативных качеств речи, языковая грамотность коммуникантов, а также многое другое.
\end{itemize}

Кроме того, сама реализация процесса общения иногда вносит в него коррективы: могут возникнуть стимулы к дальнейшему развертыванию высказывания, а могут перемениться цели, например, в результате неожиданной реакции партнера и т. д.
Могут поменяться обстоятельства --- неожиданная информация извне значительно сократит или увеличит время общения, приведет к изменению места и~т.~д.

Любой компонент речевой коммуникации может повлиять на ее процесс осуществления и на ее результат.
Все это приводит к тому, что конкретную речь можно создать и всесторонне оценить только относительно конкретных условий общения. Поэтому общение описывается как конкретная к\textbf{оммуникативная ситуация,} в которой действуют \textbf{партнеры} по общению (коммуниканты), имеющие определенные цели, обменивающиеся определенной \textbf{информацией,} используя для этого \textbf{общий код,} и действуют они в определенных \textbf{обстоятельствах.}
Эти компоненты представляют собой наиболее значимые параметры коммуникативной ситуации, которые помогают ее охарактеризовать.
Поясним, что подразумевается под каждым из них.

Общение возникает тогда, когда хотя бы у одного из партнеров появляется \textbf{потребность (и возможность) высказаться} в устной или письменной форме.
Эта потребность обычно объясняется тем, что инициатор коммуникации хочет (должен) достичь какой-либо \textbf{цели:} передать или запросить, или принять информацию, оказать воздействие и т. д.
Поэтому успешность (эффективность) общения оценивается по тому, удалось ли коммуникантам (или одному из них) достичь поставленной цели.

Очень важно, что речь всегда предполагает наличие не только ее \textbf{автора} (говорящего или пишущего), но и \textbf{адресата.}
Один из ведущих исследователей речи психолог Н. И. Жинкин даже назвал \textbf{партнеров} по общению взаимодополняющими, «как правое и левое».

В процессе коммуникации происходит передача и прием информации посредством того или иного \textbf{кода.}
Код в обшении может представлять собой не только один из национальных языков --- код может быть жестовым, цифровым, цветовым и т. д.
Главное, чтобы эта система знаков была общей для всех коммуникантов в данной ситуации, поэтому на первый план в этом отношении выходит умение кодировать, декодировать, перекодировать и хранить информацию.
Основой для любых действий с информацией является «универсальный предметный код», как его называет Н. И. Жинкин, или «язык интеллекта», универсальный потому, что он «свойствен человеческому мозгу и обладает общностью для разных человеческих языков», обеспечивая переводимость сообщений.

\textbf{Обстоятельства} общения можно условно разделить на \textbf{внутренние} и \textbf{внешние:}

\begin{itemize}
    \item внутренние: мотив, причина(ы), вызывающая это общение, психологическое или эмоциональное состояние коммуникантов;
    \item внешние: место, время, продолжительность, наличие или отсутствие непосредственного контакта и другие значимые условия общения.
\end{itemize}

При этом в зависимости от сферы общения, от особенностей ситуации каждый компонент ситуации общения может стать первостепенным.

Таким образом, все компоненты коммуникативной ситуации в совокупности и обеспечивают успешность общения на всех его этапах.
