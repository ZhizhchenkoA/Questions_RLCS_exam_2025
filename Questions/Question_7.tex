\section{Литературно-разговорная речевая культура.}

\begin{itemize}
    \item высшее и среднее образование носителя языка;
    \item рефлексивно-сенсомоторный тип интеллекта;
    \item невысокий интерес к расширению общих знаний;
    \item категоричность оценок;
    \item профессиональная деятельность требует отдельных и непостоянных, либо постоянных, но стандартных интеллектуальных усилий;
    \item удовлетворенность своим интеллектуальным багажом, отсутствие потребности в расширении своих знаний и тем более в их проверке;
    \item отсутствие постоянной привычки проверять свои знания о языке;
    \item владение преимущественно двумя функциональными стилями: обычно стилем обиходно-бытового общения (разговорной речью) и своим профессиональным стилем;
    \item частое смешение стилей в речи, неспособность их дифференцировать в общении;
    \item частое нарушение языковых норм;
    \item неспособность к развернутому монологу, даже подготовленному;
    \item стремление избежать публичной речи;
    \item преимущественно диалогический характер общения;
    \item невысокий уровень самоконтроля в процессе речи, ущербность собственной речи не осознается;
    \item «простительное» отношение к собственным речевым ошибкам;
    \item отсутствие привычки ставить под сомнение правильность своей речи;
    \item агрессия в отстаивании собственного словоупотребления: в качестве эталона обычно приводятся аргументы типа«все так говорят» или «по радио, телевидению так говорили, я слышал»;
    \item отстаивание точки зрения «главное, ЧТО сказать, а не КАК сказать»;
    \item прецедентными текстами являются средства современной массовой коммуникации и массовая литература;
    \item отсутствие осознания необходимости соблюдения коммуникативных норм речи, свободное нарушение этикетных норм общения;
    \item неумение выбрать правильную тональность общения, соответствующую изменившейся коммуникативной ситуации;
    \item частое игнорирование норм разграничения \textit{ты}- и \textit{вы}-общения и др.;
    \item переоценка своих языковых знаний, стремление к большей «литературности» речи, что при отсутствии необходимых языковых знаний приводит к искаженным представлениям о правильности;
    \item частое и неуместное употребление терминов, злоупотребление книжными и иностранными словами;
    \item допустимость жаргонных и ненормативных слов в различных коммуникативных ситуациях;
    \item небольшой словарный запас;
    \item неспособность к синонимическому варьированию речи, что приводит к штампованности и отсутствию индивидуальности в речи;
    \item экспрессия речи достигается в основном использованием категоричных оценок, сниженной лексики, громкости или интонационной напряженности артикуляции.
\end{itemize}
