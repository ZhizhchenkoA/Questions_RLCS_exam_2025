\section{Культура устной речи.}

 \textbf{Культура речи} – владение нормами устного и письменного литературного языка (произношение, ударение, словоупотребление, грамматика, стилистика), а также умение использовать выразительные средства языка в различных условиях общения в соответствии с целями и содержанием речи. Т.е. культура речи – умение правильно говорить и писать, употреблять слова и выражения в соответствии с целями и ситуацией общения. На основе этого можно выделить два критерия культуры речи – правильность и коммуникативная целесообразность с точки зрения современной эпохи. С древних времен критерии культуры речи следующие.
\begin{itemize}
    \item \textbf{Правильность} – соблюдение языковых норм (произносительных, грамматических, стилистических) – первая ступень речевой культуры.
    \item \textbf{Коммуникативная целесообразность} – обязательное представление говорящего о стилистических градациях слов и выражений, их употребление в соответствующих ситуациях.
    \item \textbf{Точность высказывания} – состоит из двух частей: точности в отражении действительности (правдивая или ложную информацию пытается донести говорящий) и точность выражения мысли в слове (донесение мысли без искажений).
    \item \textbf{Логичность изложения} – высказывание должно отражать логику действительности и мысли в логике речевого выражения. Логичность мысли (содержание высказывания) – верность отражения фактов действительности и их связей (причина-следствие, сходство-различие), обоснованность гипотезы, аргументов.
    \item \textbf{Ясность и доступность изложения} – понятность речи ее адресату. Она достигается через точное и однозначное употребление слов, терминов, словосочетаний, грамматических конструкций. Доступность (доходчивость) изложения – способность данной речи быть понятной адресату, заинтересовать его. Доходчивость предполагает ясность.
    \item \textbf{Чистота речи} – отсутствие чуждых литературному языку элементов (слов, словосочетаний) или элементов безнравственных. К ним принадлежат: слова-паразиты, диалектизмы и просторечие, варваризмы, жаргонизмы, вульгаризмы.
    \item \textbf{Выразительность речи} – такие особенности структуры речи, которые поддерживают внимание и интерес слушателей (читателей). Выразительность: информационная (слушателей интересует сообщаемая информация); эмоциональная (слушателей интересует способ изложения, манера исполнения).
    \item \textbf{Разнообразие средств выражения} – активное использование большого лексического запаса, синонимов.
    \item \textbf{Эстетичность }– реализуется в неприятии литературным языком оскорбительных для личности средств выражения смысла. Эстетика достигается с помощью эвфемизмов – эмоционально нейтральных слов, употребляемых вместо слов и выражений нетактичных (ребенок испачкал пеленки).
    \item \textbf{Уместность} – такой подбор и организация языковых средств, которые делают речь отвечающей целям и условиям общения. Уместность использования средств зависит от: контекста, ситуации, психологических характеристик личности. Суть принципа: в доме повешенного не говорят о веревке.
\end{itemize}

\subsection*{Особенности устной речи:}
\begin{enumerate}
    \item Устная речь предполагает наличие собеседника
    \item Важна реакция слушателя
    \item Говорящий создаёт свою речь "на глазах", то есть у него нет шансов начать говорить заново, если что-то показалось неудачным
    \item Избыточность устной речи не всегда оправданные повторы, уточнения, перифразы (использование синонимичных структур), которые связаны с тем, что говорящий не всегда четко может сформулировать мысль, вследствие чего ему приходится повторяться, добиваясь желаемого результата.

    \item Лаконизм (экономия) речи связан с тем, что говорящий пропускает слова, предложения, которые легко восстанавливаются слушающим из контекста.

    \item Самоперебивы, срывы конструкции такая особенность уст ной речи, когда говорящий, не закончив начатое предложение, бросает его, начинает снова, пытаясь сформулировать фразу другими словами.
    
    \item Прерывистость связана с тем, что говорящий делает неуместные паузы, так как обдумывает, что сказать, формулирует, подбирает слово. Иногда такая пауза заменяется нечленораздельны ми звуками типа мычания, прищелкиванием пальцев, восклицаниями: «Как лучше сказать?», «В общем-м-м...», «Короче говоря...» и т. д.
    
    \item Наличие невербальных (несловесных) средств коммуникации. 
\end{enumerate}
