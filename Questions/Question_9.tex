\section{Просторечная речевая культура.}

\begin{itemize}
    \item низкий уровень общего образования;
    \item работа не требует систематических интеллектуальных усилий;
    \item сенсомоторный интеллект, ориентация в общении и культуре исключительно на свою группу общения;
    \item крайняя категоричность в оценках, в целом высокая оценочность речи, оценки выражаются грубо или нецензурно;
    \item самоуверенность, безапелляционность в общении;
    \item абсолютное доминирование точки зрения«главное, ЧТО сказать, а не КАК сказать»;
    \item нежелание и неспособность следить за своей речью, контролировать ее тематически и стилистически;
    \item владение только просторечным стилем устного общения;
    \item невладение письменными формами речи; необходимость письменной речи ставит их в тупик;
    \item тексты могут писать преимущественно под диктовку;
    \item официальные документы могут писать только по образцу, заполняя пустые графы, при этом, как правило, требуют уже заполненный другими образец;
    \item отсутствие представлений о языковых нормах и языковых табу, непонимание того, что есть запрещенная к употреблению лексика;
    \item привычное, «связочное» использование вульгаризмов, жаргона, сленга;
    \item привычное использование мата в экспрессивной и связочной функции;
    \item доминирует исключительно \textit{ты}-общение;
    \item доминируют обращения типа \textit{Димон}, \textit{Вован}, \textit{Серый}, клички и под.;
    \item неспособность к чтению более или менее длинных текстов любого жанра, неспособность синтезировать смысл текста, требование объяснить им смысл устно;
    \item неумение и нежелание пользоваться словарями; агрессивное уклонение от их использования;
    \item непонимание подтекста в художественном тексте, в пословицах и поговорках;
    \item предпочтение отдается ситуативному юмору или относящемуся к телесному низу, юмористический подтекст не воспринимается, если нет грубых или нецензурных слов в рифму;
    \item лексика и фразеология, новые слова усваиваются исключительно из непосредственного общения с ближайшим окружением;
    \item отсутствует рефлексия о языке, проблемы языка не вызывают никакого интереса;
    \item в речи (как и в сознании) отсутствуют прецедентные тексты;
    \item погоня за языковой модой, тяга к модным экспрессивным словоупотреблениям;
    \item неспособность к синонимическому варьированию речи, что приводит к штампованности и отсутствию индивидуальности в речи;
    \item экспрессия речи достигается исключительно использованием категоричных безапелляционных оценок, грубой и нецензурной лексики, повышением громкости или интонационной напряженности артикуляции.
\end{itemize}
