\section{Элитарный тип речевой культуры.}

\begin{itemize}
    \item высшее образование носителя языка, обычно гуманитарное;
    \item рефлексивный интеллект;
    \item логичность мышления;
    \item некатегоричность в оценках;
    \item выполнение работы, постоянно требующей определенных интеллектуальных усилий;
    \item неудовлетворенность своим интеллектуальным багажом, наличие постоянной потребности в расширении своих знаний и их проверке;
    \item неукоснительное соблюдение этических норм общения, уважение к собеседнику и вообще к людям;
    \item владение речевым этикетом и соблюдение его норм во всех стандартных коммуникативных ситуациях;
    \item соблюдение норм литературной речи;
    \item отсутствие самоуверенности в характере и поведении;
    \item отсутствие языковой самоуверенности (отсутствие уверенности о том, что он с его точки зрения в достаточной степени уже владеет языком, его языковые знания вполне достаточны и не требуют коррекции);
    \item владение всеми функциональными стилями устной речи;
    \item незатрудненное использование функционального стиля и жанра речи, соответствующего ситуации и целям общения;
    \item «неперенос» того, что типично для устной речи, в письменную речь, а того, что свойственно письменной речи, в устную;
    \item способность контролировать свою речь в ее процессе(тематический и стилистический самоконтроль);
    \item умение выступать публично, в том числе без длительной подготовки,
    \item отсутствие боязни публичного выступления;
    \item привычка проверять свои языковые знания, пополнять их по авторитетным словарям и справочникам, спрашивать у специалистов;
    \item отсутствие автоматического подражания услышанному по радио или телевидению, прочитанному в газетах;
    \item отсутствие подражания своему непосредственному речевому окружению, самостоятельность в формировании собственной речевой культуры;
    \item богатство как активного, так и пассивного словаря;
    \item использование синонимов в речи;
    \item как минимум пассивное владение основными достижениями мировой и национальной культуры;
    \item знание прецедентных текстов, имеющих общекультурное значение, понимание их в тексте и использование их в общении;
    \item способность к логичной и последовательной устной и письменной речи;
    \item владение эпистолярным жанром, умение развертывать свою мысль как устно, так и письменно;
    \item умение приводить несколько аргументов в дискуссии, обсуждении, споре;
    \item способность к языковой игре (игре со словом), умение и уместность ее использования, получение удовольствия от языковой игры окружающих и собственной языковой игры;
    \item умение использовать сниженную лексику и фразеологию экспрессивных, художественно-изобразительных целях;
    \item понимание речевого юмора, получение удовольствия от речевого юмора;
    \item умение воспринять подтекст в шутке, анекдоте, пословице, поговорке;
    \item умение самостоятельно воспринять подтекст в художественном тексте большого объема;
    \item умение оценить как форму, так и содержание воспринятого текста;
    \item преимущественное использование формы вы-общения, тщательное соблюдение нормы употребления ТЫ и ВЫ, отсутствие в речи общеупотребительных штампов;
    \item получение удовольствия от восприятия сложных текстов
    \item теоретических дискуссий;
    \item нелюбовь к примитивным диалогам в вербально и визуально воспринимаемых текстах;
    \item фиксация речевых ошибок в устной и письменной речи окружающих, в письменных и медийных текстах, в рекламе;
    \item обсуждение с коллегами и близкими состояния современного языка, комментирование изменений в языке.
\end{itemize}
