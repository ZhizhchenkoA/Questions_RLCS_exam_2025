\section{Аргументирующая речь}  
\textbf{Аргументирующая речь} - один из самых распространенных типов речи. В риторике к нему относят как убеждающую, так и агитирующую речи, поскольку они имеют много общего. Коммуникативная цель говорящего в этом случае - убедить собеседника в правильности какого-то положения, заставить его изменить свои взгляды, мнения (убеждающая речь), а так же уговорить его на что-то, склонить к какому-либо действию (агитирующая речь). Аргументирующая речь связана с такими понятиями, как тезис, аргументы, демонстрация. Без осмысления этих понятий нельзя говорить о структуре и особенностях аргументирующей речи. Аргументирующая речь обладает следующими свойствами: 
	- наличием четко поставленной коммуникативной цели - убедить собеседника (адресата);
	- адресат в аргументирующей речи, как правило, не остается лишь объектом воздействия. Он представляет собой субъекта с собственной позицией, активно ее отстаивает;
	- основной содержательный элемент речи - аргументы, доказательства, приводимые говорящим в защиту собственной позиции, и контраргументы - доказательства в защиту противоположной позиции, приводимые адресатом;
	- противоположность позиции в аргументирующей речи легко может видоизмениться и перерасти в конфликт;
- реализации поставленной цели может помешать наличие уже готового решения по данному вопросу у адресата.

\chapter{Как формулируется тезис?}

\textbf{Тезис} - мысль или положение, истинность которого требуется доказать. Тезис должен быть истинным, иначе никакими доказательствами не удастся его обосновать. Кроме того, тезис необходимо ясно, понятно, точно сформулировать, следить, чтобы он не содержал логического противоречия. Тезис не должен меняться на протяжении всей речи. Если тезис будет нечетко сформулирован, его трудно будет доказать, у слушателей непременно возникнут вопросы: \textit{«Ну и что это доказывает? Это совсем из другой оперы»} . Нечеткость формулировок может привести к потере тезиса. Это очень частая ошибка: начав говорить, оратор может потерять нить выступления, сбиться на второстепенные или даже не имеющие отношения к теме проблемы, начав говорить \textit{«вокруг да около»}.
    
    \chapter{\textbf{Какие виды аргументов целесообразно использовать?} } 
	Сформулировать тезис в соответствии с требованиями логики еще не все. Самое главное в том, чтобы тезис был доказан. Для доказательства используют аргументы - суждения или совокупность суждений, приводимые для подтверждения другого суждения (тезиса). Как и доказываемый тезис, аргументы должны быть истинными и достаточными, при этом они не могут существовать независимо друг от друга, а объединяются в особую систему утверждений. Среди аргументов можно выделить аргументы разных типов:

- \textbf{Сильные и слабые аргументы}. Довод, против которого легко найти возражение, называют слабым, и наоборот, довод, который трудно опровергнуть, называют сильным. Сила аргумента определяется с точки зрения слушателя, а не говорящего. Так для родителей, чей ребенок подстригся несколько необычно, слабым аргументом будет \textit{«Всем ребятам в классе понравилось!»}, а сильным - \textit{«Директор сказал, что эта прическа идеальна для ученика»}. Для того, чтобы убедить, лучше оперировать сильными доводами. Бывают случаи, в которых достаточно привести один - исчерпывающий- аргумент, но такие ситуации редки. Есть мнение, что оптимальное количество аргументов - 3-4, поскольку один аргумент - это просто факт, на два аргумента \textit{«во-первых, во-вторых»} - можно возразить, на три аргумента это сделать сложнее, а после четвертого аргументы воспринимаются как \textit{«много»}, что тоже нельзя рассматривать как положительный факт. В этом \textit{«много»} теряется вес каждого конкретного аргумента, аудитория перестает следить за развитием логики доказательства, поскольку не в состоянии охватить весь объем материала.

- \textbf{Аргументы \textit{«к делу»} и \textit{«к человеку»}}.Первый тип аргументов(это в первую очередь факты, подтверждающие справедливость выдвинутого тезиса, а так же законы природы, аксиомы - все то, что может рассматриваться как безусловно истинное, не требующее отдельного доказательства) называют еще рациональными, воздействующие на разум слушателей, второй - иррациональными, психологическими, вызывающие к чувствам. Психологические аргументы могут быть двух видов. Во-первых, они действительно могут апеллировать к личностным качествам оппонента: \textit{«Тебе «Титаник» не понравился? Нет? Да что ты понимаешь, ты в жизни кроме мультиков и ужастиков ничего не смотришь!»} или, во-вторых, к чувствам аудитории:   \textit{«Вы, как люди высокообразованные и интеллигентные, не можете не понимать значения изучения риторики в вузе»}. Подразумевается: если не видишь смысла изучать риторику, значит, ты не интеллигентный и не высокообразованный человек. Вместе с тем логические и психологические доводы часто выступают в единой системе, которая и называется системой аргументации.

- \textbf{Собственно аргументы и контраргументы}. Последние возникают в сознании оппонента как возражения оратору. Очень хорошо, если говорящий зная или чувствуя аудиторию, в состоянии предвидеть предусмотреть возможные контраргументы, а еще лучше - предупредить их появление фразами: \textit{«Здесь мне могли бы возразить, что...но это не совсем так, потому что...»}.
Продумать формулировку тезиса и систему аргументов крайне важно, так как воздействие этих аргументов будет определяться еще и тем, как именно они расположены и  в какой последовательности они представлены слушателям.

\chapter{\textbf{Как правильно расположить аргументы?}}

1.\textbf{Нисходящая и восходящая аргументация}.
Различие этих способов в направлении вектора усиления аргументации. При \textbf{нисходящей} аргументации сначала приводятся самые сильные аргументы, затем - менее сильные, завершается все выводом или побуждением к действию. При нисходящей аргументации слабые аргументы выступают как дополнение к сильным, а не как самостоятельные, выглядят более убедительно.
\textbf{Восходящая} аргументация предполагает, что сильные аргументы используются в конце выступления. Заканчивается речь с восходящей аргументацией, как правило, ярким, эмоциональным призывом.

2. \textbf{Односторонняя и двусторонняя аргументация}. \textbf{Односторонняя}  аргументация предполагает изложение аргументов одной направленности: либо позитивные, поддерживающие данную точку зрения, либо негативные, излагающие противоположную точку зрения. При двусторонней аргументации аудитория может сопоставить обе позиции и вместе с оратором выбрать одну из них. Оппонент при двусторонней аргументации может быть как реальным, так и вымышленным лицом. Последний прием называется \textit{«чучелом оппонента»}. Двусторонняя аргументация активизирует внимание аудитории, позволяет более глубоко проанализировать ситуацию и сделать выводы самостоятельно. Естественно, что этот прием предпочтителен в аудитории более высокого интеллектуального уровня. Односторонняя аргументация лучше воспринимается слушателями, готовыми принять чужую точку зрения, особенно если она совпадает с их собственной.

3. \textbf{Индуктивная и дедуктивная аргументация}. При использовании индуктивной аргументации изложение построено от анализа конкретных фактов к выводу. Таким образом, обобщение делается на основе изучения типичных случаев. При этом выступающему необходимо следить, чтобы все примеры, рассмотренные в качестве основания для обобщения, были истинными и связанными с выводами, а их количество должно быть достаточным для соответствующих обобщений. Кроме того, необходимо, чтобы эти примеры были типичными для подтверждения сделанного вывода. Особое внимание оратору следует обратить на наличие отрицательных, как говорит П. Сопер, примеров — контр-доводов, не подтверждающих основной вывод. Не исключено, что если о наличии таких отрицательных примеров не вспомнит докладчик, то может вспомнить кто-нибудь из слушателей, и тогда вся речь потеряет свою убедительность. Особенно плохо, если оратор и не подозревал о наличии такого рода отрицательных примеров, если они явились для него неожиданностью. Это лишний раз свидетельствует о том, что к выступлению надо готовиться серьезно, быть, как говорят журналисты, \textit{«в материале»}. Дедуктивный метод предполагает рассуждение от общего к частному. Познавая мир, человеческое общество использует этот метод постоянно, поскольку ряд явлений, которые общество уже постигло, приобретает значение общего правила. Это правило становится следующим общим положением для следующего познавательного процесса, и так продолжается бесконечно. У нас не вызывает сомнения факт смены времен года в определенной последовательности — это общее положение, которое ложится в основу силлогизма, используемого в дедукции: \textit{«За весной приходит жаркое лето. Сейчас весна. Следовательно, нужно готовить легкую одежду»}. Если говорить о построении речи в целом с использованием только метода дедукции, то следует отметить, что такие речи не очень хорошо воспринимаются аудиторией, так как в них чувствуется назидательность, навязывание мнения выступающего.

\chapter{\textbf{Как может использоваться аргументирующая речь в бытовом общении?}}
Аргументирующей речью мы часто пользуемся в бытовом общении, когда просим родителей профинансировать какую-то покупку, разрешить что-то сделать, отпустить куда-либо. При чем, если просьба простая, мы, не задумываясь и не рассчитывая на отказ, просто просим или информируем, одновременно ожидая одобрения: 
- \textit{Мам, дай полтинник. У девочки в группе день рождения. Мы хотели ей цветы купить}. 
- \textit{Я зайду сегодня после занятий к Свете, ладно?}
Если просьба серьезная, то получение денег требует хорошего обоснования. Говорящий осознает цель, понимает, что необходимо серьезно, используя убедительные аргументы обосновать свою просьбу. Другими словами, нужно создать аргументирующую речь,
аргументирующий текст. Такая речь редко произносится спонтанно, если проблема серьезна и значима. Вероятнее всего, ее хорошо обдумают. 