\section{Неоэпистолярная культура}


Рубеж веков охарактеризовался технологическим прорывом, внесшим серьезные изменения в жизнь современного человека. Повсеместное распространение новых технических средств коммуникации резко сократило физическую дистанцию между участниками общения и во многих случаях позволило не учитывать наличие географических и политических границ между коммуникантами.

Если еще недавно специалисты-языковеды указывали на умирание эпистолярного жанра, утверждая, что телефонная связь вытеснила обмен письмами, то в конце второго десятилетия XXI века частотное употребление выражений «напиши мне», «скиньте мне информацию на почту», «жду письма» и ставшая обязательной информация об адресе электронной почты в ситуации закрепления знакомства и организации коммуникационных каналов свидетельствуют о повсеместном и всеобъемлющем распространении общения в письменной форме.
Однако принципиально новые каналы обмена письменными текстами,
позволяющие мгновенно доставить письмо (здесь и далее термин «письмо» используется в значении «зафиксированный при помощи любых технических средств и воспринимаемый визуально текст, посылаемый кому-либо для какого-либо сообщения») адресату, находящемуся на любом расстоянии от адресанта, заставляют говорить о новом этапе существования письменного общения. Новые условия предполагают появление новых эпистолярных норм и позволяют говорить о \textbf{неоэпистолярной культуре.}

В процессе исследования современных норм эпистолярного общения
на материале нескольких языков (русского, английского, французского, немецкого, греческого и др.) были выявлены общие тенденции, характерные для участников общения, принадлежащих к разным языковым культурам. 
Изучению подверглись как эксплицитно выраженные нормы, сформулированные в различных руководствах по письменной коммуникации для носителей языка и в учебных пособиях для иностранцев, так и корпусы эпистолярных текстов, созданных носителями разных языковых культур, при условии что адресат и адресант являются носителями одного и того же языка и одной и той же культуры.

Наиболее устойчивые зоны эпистолярного текста – \textbf{формулы приветствия и прощания – максимально информативны и демонстративны} с точки зрения изучения норм эпистолярной коммуникации. Малейшие изменения, встречающиеся и (или) закрепляющиеся в инициальной и финальной формулах, свидетельствуют о серьезных изменениях в эпистолярном этикете.

Сравнение исследуемого современного эпистолярного материала с аналогичными источниками информации, датированными концом XX – нач. XXI вв., позволяет делать вывод не только о минимализации пространственной и временнОй дистанции между адресантом и адресатом, но и о сокращении
иерархической дистанции между коммуникантами. На материале каждого из исследованных языков наглядно прослеживается тенденция \textbf{использования в неоэпистолярной культуре этикетных формул, характерных для недистанцированного общения} (например, предфинальное «на связи!» вместо «жду
Вашего письма»). 

Сфера официально-делового эпистолярного общения сближается со сферой личных коммуникаций за счет \textbf{широкого использования
этикетных формул}, характерных для последней, и вытеснения в маргинальную (периферийную) зону тех формул, которые были доминантными для
деловой сферы еще 15–20 лет назад

Таким образом, в настоящее время можно говорить о \textbf{сокращении не только физической, но и этикетно-иерархической дистанции между адресантом и адресатом}. При этом общность тенденций, характерных для неоэпистолярной культуры разных европейских народов, свидетельствует о глобальном характере этих тенденций и большей транспарентности этикетных норм
межкультурной коммуникации в современных условиях.
