\section{Письмо как вид речевой деятельности.}


\textbf{Письмо (письменная речь) -- вид речевой деятельности, в процессе которого содержание высказывания передается с помощью графических знаков.} 

Результатом письма как вида речевой деятельности является письменное высказывание, письменный текст. 
Письмо является одним из способов реализации \textit{опосредованного общения}, когда общающиеся не только не видят и не слышат друг друга, но и в ряде случаев незнакомы друг с другом. 
Так, автор статьи, книги, стихотворения, создавая речевое произведение, ориентируется на читателя вообще, а не на конкретного человека. 

Письменная речь это, как правило, подготовленная речь. Исключением являются такие жанры обиходно-бытового общения, как записка, дневниковые записи, письма личного характера, создание которых не требует большой предварительной подготовки. Подготовленность письменной речи, возможность ее многократного и всестороннего обдумывания определяют ее основные особенности.
Письменная речь реализуется, как правило, в форме монологического высказывания. 
Письменное монологическое высказывание рассчитано на зрительное восприятие, что позволяет адресату неоднократно перечитывать написанное, постепенно осмысливать его содержание, вдумываясь в значение слов, выражений и других компонентов текста.
Письменные высказывания, как правило, имеют \textit{сложную композиционную структуру}, что обеспечивается наличием в нем смысловых блоков, каждый из которых играет определенную роль в передаче содержания и замысла речи. 
Одним из способов оформления смысловых блоков является правильное построение, обязательное выделение и последовательное расположение абзацев, так как именно \textit{абзац} - содержательно и структурно обеспечивает поступательное, целенаправленное выражение основной мысли текста.
При оформлении письменного высказывания автор стремится передать свое эмоциональное состояние, подчеркнуть главное, высказать свое отношение к предмету речи не только словесными средствами, но и с помощью знаков препинания, графических обозначений, шрифтовых выделений и т. п.
Хотя в письменной речи реализуется опосредованное общение, авторы книг, учебников, статей стремятся использовать различные средства \textit{диалогизации}.

Письмо (письменная речь) в речевой деятельности человека занимает $9\%$. 
Однако именно письменные высказывания - книги, стихи, учебники, монографии, статьи, документы, газетные публикации играют решающую роль в развитии духовной культуры человека, обеспечивают передачу знаний, формируют общественное сознание, закрепляют нормы отношений в государстве и т. п.