\section{\textbf{Эпидейктическая речь}}
\textbf{Эпидейктическая речь} (от греч.  \textit{deiknum }- показываю, делаю видным, известным, приветсвую), -- речь, цель которой - выразить свое понимание добра и зла, прекрасного и постыдного:  \textit{«...Поговорим о добродетели и пороке, прекрасном и постыдном, потому что эти понятия являются объектами для человека, произносящего хвалу и хулу»}(Аристотель «Риторика»). \textbf{Жанры}, в которых реализуется этот вид речи, -- \textbf{поздравление, тост, похвала} - встречаются часто как в быту, так и на официальных мероприятиях: празднованиях, юбилеях, и т.д. Её основное содержание - похвала и порицание, оценка, но вместе с тем эпидейктическая речь может так же информировать, в каких-то ситуациях развлечь слушателя, содержать элементы аргументации. По Аристотелю, \textbf{задача эпидейктической речи} — хвала добродетели в разных ее проявлениях, хвала всему, что достойно похвалы: справедливости, мужеству, мудрости, щедрости, благоразумию и другим достойнейшим качествам. Параллельно с этим эпидейктическая речь, по правилам древних, могла возбуждать и другие, прямо противоположные чувства. Аристотель называет их «хула». Хула, т. е. порицание, осуждение, была направлена на возбуждение столь же сильных, но осуждающих чувств. Осуждались трусость, несправедливость, пороки, свойственные человеческой натуре, скупость и др. Но, даже порицая, эпидейктическая речь (и в этом ее особенность!) возбуждала не чувство уныния или безнадежности, а чувства негодования, гнева — словом, чувства, направленные на осуждение и, мы бы сейчас сказали, исправление недостатков. Особый вид эпидейктической речи обращен, по мнению Аристотеля, к прошлому, к истории: \textit{«Прекрасно также все памятное, и чем памятнее, тем прекраснее»}. Не случайно и до сих пор торжественные хвалебные речи часто обращены к прошлому: они произносятся в честь годовщин побед, юбилеев победителей, при открытии исторических памятников.\textbf{ Цель каждой такой речи} — возбудить в потомках чувство благодарности, гордости, чувство патриотизма и сопричастности. Чтобы решить поставленные задачи —  возбудить у аудитории высокие чувства — эпидейктическая речь регламентировалась рядом правил. Аристотель считал, что а) хвалить нужно то, что больше всего ценится в обществе и, соответственно, у данной аудитории; б) законы эпидейктической речи требуют усиления и преувеличения: оратору нужно было показать, как герой мужественно боролся с непреодолимыми трудностями, лишениями, препятствиями и как сумел все преодолеть. При этом преувеличение, как отмечает А. К. Михальская, — не неискренность, а закон жанра, принятая в речевой традиции условность; в) оратор должен использовать сравнение действий восхваляемого либо с поступками героев прошлого, либо с действиями обычных людей, чтобы выделить на этом фоне его заслуги; г) случайно совершенный поступок древние представляли как осуществленный обдуманно. Традиции произнесения эпидейктических речей на Руси складываются из памятников красноречия \textbf{дидактического}(учительского, т.е. поучение, беседа. Например, "Поучение Владимира Мономаха") и памятников \textbf{красноречия торжественного}(более глубокое, чем дидактическое, требовало знаний общей культуры, особого мастерства и тщательной работы, большого литературного мастерства. Например, \textit{«Слово о полку Игореве»}). Для торжественной речи характерны обращение к читателю или слушателю, риторические вопросы, восклицания, метафоры, антитеза и повтор. \textbf{Композиция эпидейктической речи была обычно трехчастной}: вступление, повествовательная часть, заключение. \textbf{Вступление} - относительно самостоятельная обязательная часть, цель которой - привлечь внимание слушателей и обозначить проблему, которую ставит автор. \textbf{Основная часть речи} - повествовательная - представляет собой рассказ о том или ином конкретном событии. Рассказ перемежается лирическими отступлениями, комментариями автора, пояснениями. \textbf{Заключение} - завершающая часть: похвала герою; обращение к читателю или слушателю с призывом; молитва.\textbf{ К протокольно-этикетным жанрам} относятся: приветствия делегаций, речь при официальной встрече гостей, официальное поздравление юбиляра, траурная речь, речь с оценкой заслуг коголибо и др. \textbf{Протокольно-этикетное выступление} должно быть кратким, эмоциональным, не содержать ничего спорного, дабы не возбудить несогласия, только однозначные утверждения, создавать впечатление понятности ситуации и решаемости всех проблем. Как правило, речи такого типа подчеркивают значение чего-либо в общественной жизни коллектива, региона или страны, сообщества стран. Оратор должен быть настроен оптимистично, дружелюбно, произносить речь уверенно, энергично (за исключением тех случаев, когда произносится траурная речь). Таким образом, несмотря на все разнообразие жанров, эпилейктическая речь обладает общим комплексом содержательных и структурных элементов, а главное — \textbf{общей целью -- возбуждение у слушателей чувств и эмоций}. 
