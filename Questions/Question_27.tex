\section{Подготовка публичного выступления. Характеристика этапов подготовки публичного выступления.}
Этапы реторического канона:
\begin{itemize}
	\item \textbf{инвенция} - изобретение темы и идеи
	\item \textbf{диспозиция} - план
	\item \textbf{элокуция} - расписывание текста
	\item \textbf{меморио} - репетиция и запоминание
	\item \textbf{акцио} - выступление
\end{itemize}

\subsection{Инвенция}
Необхордимо cформулировать тему, определить перечень вопросов и степень их важности для раскрытия темы, выделить рему, поддерживающую обоснование темы

\subsection{Диспозиция}
План обязательно должен включать в себя вступление, основную часть и заключение. При
этом на вступление и заключение отводится по 10 процентов регламента.
\textbf{Этапы написания плана:}
\begin{itemize}
	\item предварительный;
	\item рабочий после того как изучена необходимая литература, собран фактический материал
	\item основной
\end{itemize}

\subsection{Элокуция}

\textbf{Принципы логико-композиционного построения ораторской речи:}
\begin{itemize}
	\item \textbf{принцип последовательности} - каждая вновь высказанная мысль должна вытекать из предшествующей
	\item \textbf{принцип усиления} - значимость и убедительность аргументов должны постепенно нарастать, самые сильные доводы приберегаются к концу выступления
	\item \textbf{принцип экономии} - поставленная цель должна достигаться наиболее простыми рациональными способами с минимальными затратами усилий, времени и речевых средств
\end{itemize}

\subsubsection{Вступление}
Цель которой состоит в том, чтобы привлечь внимание аудитории. Опытные ораторы рекомендуют начинать с интересного примера, пословицы, поговорки, крылатого выражения, юмористического замечания. Во вступлении можно использовать цитату. Не следует начинать выступление непосредственно с существа вопроса, потому что аудитории требуется несколько минут, чтобы
привыкнуть к тембру голоса оратора, манере его поведения. Именно по этой причине опытные ораторы тратят первые несколько минут на то, чтобы поблагодарить председателя, объявившего их выступление. Однако в начале речи не стоит приносить извинений за то, что вы не готовы, что недостаточно компетентны, что вообще
взяли слово.
\subsubsection{Основная часть}
В основной части важно сохранить логическую последовательность и стройность в изложении материала. Существуют различные методы его преподнесения:
\begin{itemize}
	\item \textbf{индуктивный метод} - от частного к общему. Оратор начинает речь с конкретного случая, а затем подводит слушателя к
	обобщениям и выводам. Этот метод часто используется в агитационных выступлениях;
	\item \textbf{дедуктивный метод} - от общего к частному. Сначала
	оратор выдвигает какое-либо положение, затем разъясняет его
	смысл на конкретных примерах. Данный метод применяется в выступлениях пропагандистского характера;
	\item \textbf{метод аналогии} - сопоставление различных явлений,
	фактов, событий с тем, что хорошо известно слушателю;
	\item \textbf{концентрический метод} - расположение материала вокруг главной проблемы, поднимаемой оратором (в речи всегда присутствует центральная проблема и круг более частных проблем, которые рассматриваются в связи с центральной);
	\item \textbf{ступенчатый метод} - последовательное изложение одного вопроса за другим, без возвращения к предыдущему;
	\item \textbf{исторический метод} - изложение материала в хронологической последовательности.
\end{itemize}

В зависимости от типа речи содержание основной части может быть различно:
\begin{itemize}
	\item \textbf{Аргументирующий тип}
	
	Тезис должен быть истинным, не должен содержать логических противоречий, тезис должен оставаться постоянным на всём протяжении речи. Аргументация может быть:
	\begin{itemize} 
		\item восходящей: главный минус - не все слушатели дождутся сильных аргументов
		\item нисходящей: главный минус - остаётся внутренняя дискуссия
		\item кольцевой
	\end{itemize}
	
	\item \textbf{Информирующий тип}
	
	Основная часть должна включать:
	\begin{itemize}
		\item анализ конкретных фактов
		\item нерешённые проблемы
		\item сопоставление нового и старого
		\item тематическая однородность
	\end{itemize} 
	
	\item \textbf{Эпидеиктический тип}
	
	Основная часть - повествование о каком-то событии с комментариями от авторами, примерами, лирическими отступлениями и т.п. Приветствуется использование художественных преувеличений.
\end{itemize}

\subsubsection{Заключение}

В заключении рекомендуется повторить основные
мысли, суммировать наиболее важные положения и сделать краткие
выводы. Убедительное и яркое заключение всегда запоминается
слушателям. Недопустим обрыв речи по причине нарушения регламента. Плохо, если оратор заканчивает речь в стиле самоуничижения.
Конец должен быть таким, чтобы слушатели почувствовали, что
дальше говорить нечего. Последние слова оратора призваны воодушевить слушателей или призвать к активной деятельности. Необходимо чётко дать понять, что оратор завершил своё выступление.